\chapter{Conclusions}
\label{chap:conclusions}

Monte Carlo particle transport methods are being considered as a viable option
in the future for high-fidelity simulation of nuclear reactors. This is, in
part, thanks to the enormous advances in high performance computing. If parallel
methods were not exploited, the sheer number of floating point operations
(FLOPs) required to reduce stochastic uncertainty to acceptable levels when
using Monte Carlo would require unacceptably long simulation time. However, the
present availability of supercomputers with hundreds of thousands or millions of
processor cores means that, solely from the perspective of FLOPs required,
solution to LWR problems using Monte Carlo could hypothetically be done in a
short time.

Lest we be fooled into thinking floating point operations are the only challenge
to overcome, let us remind ourselves that they are but one small piece of the
puzzle --- there are a variety of algorithmic and architectural challenges that
will not be easy to solve; an excellent summary of these issues, especially as
they apply to full-core LWR calculations, has recently been given by Martin
\cite{net-martin-2012}. The objective of this thesis was to make headway towards
overcoming some of these challenges.

In \autoref{chap:intro}, an overview of four contemporary issues preventing
Monte Carlo simulations of realistic LWR models was given: source convergence,
cross section memory requirements, tally memory requirements, and degradation in
parallel efficiency. In the following sections, we will summarize the status of
and progress that has been made in each area as well as what ongoing work will
still be needed in order to reach the goal of performing reactor analysis with
Monte Carlo.

\section{Source Convergence}

One area that has received quite a bit of attention in the last decade is source
convergence in Monte Carlo calculations. There are, in a sense, two separate
issues relating to source convergence. The first is being able to assess the
convergence of the source distribution, i.e. because the source distribution is
a collection of finite points in multiple dimensions, how can one assure that it
has reached stationarity in the method of successive generations. For a scalar
value, such as the global eigenvalue, it is intuitive to merely look a line plot
of its value versus the number of batches.

The second issue is accelerating convergence of the source distribution. Until
the source has converged in a Monte Carlo eigenvalue calculation, it is not
possible to accumulate tallies. Thus, the inactive batches are ``wasted'' as it
were. If a method could be employed to reduce the number of inactive batches, we
would save the wasted calculational time. Both issues are made worse by a high
dominance ratio; this is unfortunately the case in a full-core reactor where the
dominance ratio can be very close to unity.

\subsection{Status}

The pioneering work of Ueki and Brown \cite{trans-ueki-2002, physor-brown-2006}
has helped introduce and gain acceptance of the Shannon entropy, a scalar metric
based on information theory, as a proper means for assessing convergence of the
source distribution. Shannon entropy has now been implemented in most modern
Monte Carlo codes, including OpenMC (see \autoref{sec:shannon-entropy}). While
Shannon entropy will work for nearly all models, it may be ill-suited for very
large, loosely-coupled models such as a spent fuel array; for these problems,
research efforts are looking into mesh-based convergence diagnostics
\cite{mc-she-2011}.

Various methods have been proposed, tested, and implemented in Monte Carlo codes
for accelerating source convergence. In the author's opinion, one of the more
promising methods, which is widely employed in deterministic calculations, is
coarse mesh finite difference. The basic idea is to use volume-integrated
reaction rates and partial currents on mesh surfaces from Monte Carlo tallies to
solve a low-order system. This method has been proven to be quite effective in
reducing the number of inactive batches in Monte Carlo eigenvalue
calculations \cite{physor-lee-2012}.

We see then that significant progress has been made in the area of source
convergence in Monte Carlo eigenvalue calculations. For full-core LWR models,
CMFD can be used to accelerate source convergence, and Shannon entropy can be
used to assess source convergence. As a result, it was not an objective of this
thesis to pursue further work in the area of source convergence.

\subsection{Future Work}

Nevertheless, there are still a number of interesting areas of research
pertaining to source convergence. While source convergence can be assessed using
Shannon entropy, there is as-of-yet no viable method for diagnosing
undersampling. Undersampling is a phenomenon whereby local tallies may be
systematically biased as a result of not using enough particles per batch
\cite{nse-ueki-2005, nse-ueki-2008}. To the author's knowledge, only one method
for undersampling diagnosis has been proposed in the literature
\cite{jnst-ueki-2011}, but it unfortunately would not be feasible in a realistic
calculation due to memory requirements. Moreover, while the bias in global
eigenvalue is fairly well understood \cite{ane-brissenden-1986} and known to be
small, no study has systematically looked at bias in local tallies due to
undersampling.

In the area of source convergence acceleration, one current ``hot topic''
pertaining to CMFD is determining the best way to generate multigroup parameters
for the diffusion solver. In particular, generating multigroup diffusion
coefficients from a Monte Carlo code can be problematic. A number of research
efforts are currently looking at methods for generating diffusion coefficients
in the hope that more accurate parameters will lead to better convergence
properties \cite{ane-fridman-2011, nse-pounders-2009}. Researchers are also
looking at other low-order methods for accelerating convergence.

\section{Parallel Efficiency}

Solution of full-core LWR problems will require an enormous number of floating
point operations and will thus necessitate the use of thousands or millions of
processors in parallel. However, with current parallel algorithms, a serious
degradation in parallel efficiency will likely be experienced when one tries to
use even thousands of processors \cite{physor-hoogenboom-2012}. To reiterate,
two of the root causes for this degradation are reduction of tallies and
synchronization of the fission source at each batch in an eigenvalue
calculation. One of our major objectives in this work has been to develop
algorithms that ameliorate the degradation of parallel efficiency at large
numbers of processors.

\subsection{Contributions of Present Work}

In \autoref{chap:fission-bank}, a nearest-neighbor algorithm was proposed and
subsequently implemented in the OpenMC Monte Carlo code. This algorithm takes
advantage of the fact that many of the fission sites produced on one processor
can be used as source sites on that same processor --- in doing so, it avoids
unnecessary communication between processors. A theoretical analysis of the
algorithm shows that the expected cost is $O(\sqrt{N})$, $N$ being the number of
particles per generation, whereas traditional master-slave algorithms are $O(N)$
at best, and possibly even $O(N \log_2 N)$. The algorithm was tested on two
contemporary supercomputers, the Intrepid Blue Gene/P at ANL and the Titan Cray
XK7 at ORNL, and demonstrated nearly linear parallel scaling up to 163,840
processor cores. There is no reason to believe that the nearest-neighbor
algorithm will not scale to even greater processor counts.

In \autoref{chap:tally-reduction}, an algorithm for reducing network
communication arising from tally reduction was analyzed and, again, implemented
in OpenMC. The basic idea is conceptually very simple --- the grouping of
particles histories into batches for the sake of calculating variance is
arbitrary. The proposed algorithm groups only particle histories on a single
processor and in doing so prevents all network communication for tallies until
the very end of the simulation. In a large scale parallel calculation of, say, a
realistic LWR model, it is likely that enough processors will be used to obtain
a sufficient number of realizations of the random variables. This algorithm was
tested in OpenMC on the Monte Carlo performance benchmark on a cluster at MIT,
and it was shown that network communication was substantially reduced.

Together, the algorithms presented in \autoref{chap:fission-bank} and
\autoref{chap:tally-reduction} should enable very high parallel efficiencies to
be attained for eigenvalue calculations using the method of successive
generations. These algorithms have been published in \cite{nse-romano-2012} and
\cite{trans-romano-2012}, respectively.

\subsection{Future Work}

There are a few potential shortcomings of the nearest-neighbor fission bank
algorithm. One that was mentioned in the conclusions of
\autoref{chap:fission-bank} is that the algorithm would preclude the use of load
balancing via existing algorithms for heterogeneous computer architectures. A
simple method to provide load balancing in such situations based on ``tuning''
the algorithm was suggested, but it has never been tested. Additionally, the
nearest-neighbor algorithm may be very difficult to combine with any domain
decomposition scheme since the order of fission bank sites would necessarily
depend on their spatial coordinates. If domain decomposition is ever to be used
for calculations with tens of thousands of processors or more, this issue will
likely need to be addressed.

The tally reduction algorithm is robust; we see no immediate need to pursue
further work on it. Perhaps one downside of the method is that it will not
exactly reproduce the variance of a calculation in which particle histories are
batched according to single fission generations. However, this is no different
than saying that any calculation that uses batching will not reproduce the
variance exactly. It was argued that, in the absence of intergenerational
correlation, the expected value of the variance is the same regardless of the
batching. In fact, batching particle histories across fission generations will
produce more reliable estimates of variance in an eigenvalue calculation since
intergenerational correlation effects are avoided in doing so.

\section{Cross Section Memory}

In order to perform a realistic simulation of a reactor at power, it is
important to account for the dependence of interaction cross sections on
temperature. In \autoref{sec:cross-section-memory}, this was discussed at
length, and we found that the memory requirements resulting from storing cross
sections at a multitude of temperatures could be as high as hundreds of
gigabytes. Thus, methods are needed to reduce or decompose the cross section
data. We reiterate that this thesis has not looked at this particular issue
since other research efforts have shown promise.

\subsection{Status}

Two substantial efforts are underway aimed at reducing cross section memory
requirements. The first is the work of Yesilyurt, Martin, and Brown on
on-the-fly Doppler broadening of cross sections \cite{nse-yesilyurt-2012,
  trans-brown-2012}. In this method, the cross section at any temperature is
represented as a series expansion --- thus only the series expansion
coefficients need to be stored in memory. Preliminary work on this method looks
promising, but it has yet to be demonstrated for a realistic problem with
hundreds of nuclides at many temperatures.

The other more recent research effort is an explicit temperature treatment by
Viitanen and Leppänen \cite{nse-viitanen-2012}. This is a fundamentally
different approach wherein cross sections are only stored at 0 K and the effect
of the thermal motion of the target material is accounted for by using a
rejection sampling technique. This would reduce the necessary cross section
storage to that of a single temperature. The work is still in a preliminary
state; however, it has been demonstrated rigorously that this method can account
for the thermal motion of nuclides in material at a temperature greater than 0
K.

\subsection{Future Work}

In both the on-the-fly Doppler broadening and explicit temperature methods, the
temperature dependence of $S(\alpha,\beta,T)$ data and unresolved resonance
probability tables can not be accounted for. For any realistic reactor
simulation, accounting for the temperature dependence of $S(\alpha,\beta,T)$ is
absolutely essential to obtaining accurate results. It should be noted however
that $S(\alpha,\beta,T)$ and probability table data is typically a very small
fraction of the overall cross section data. As such, it may be feasible to
simply store these data at very fine temperature intervals. However, this has
not been studied.

The explicit temperature method has been shown to produce unbiased results
\cite{nse-viitanen-2012}. However, no results in the literature have
demonstrated what the effect on the figure of merit for tallies would be,
i.e. while the absolute performance cost was shown to be reasonable
\cite{physor-viitanen-2012}, does the method require significantly more
particles to obtain comparable statistics? Until that question is answered, it
is not possible to assess the performance penalty of the method versus using
normal Doppler broadened cross sections.

\section{Tally Memory --- Domain Decomposition}

In the current SPMD parallel technique for Monte Carlo particle transport, all
problem data stored in memory must be replicated on each process in a
simulation. For the simulation of a light-water reactor, the memory requirements
for tallies may be enormous, likely exceeding terabytes. A method for
decomposing tally memory across many processors is critical in such a
scenario. In this thesis, we have looked at two different schemes for
decomposing tally memory, domain decomposition and data decomposition.

\subsection{Contributions of Present Work}

A theoretical framework for analyzing domain decomposition of Monte Carlo
particle transports has only been posed in the last year or so
\cite{jcp-siegel-2012-1}. In \autoref{chap:domain-decomp}, we presented a
theoretical analysis of domain decomposed Monte Carlo particle transport
simulations looking at the effect of load imbalances on the total simulation
time relative to a perfectly load balanced simulation\footnote{This is analogous
  to a simulation performed with no domain decomposition.}. The analysis
demonstrated that load imbalances in domain decomposed simulations arise from
two different phenomena: non-uniform particle densities and non-uniform spatial
leakage. One of the interesting facts we can glean from the analysis is that the
dominant performance penalty in domain decomposition comes not from network
communication but from these load imbalances. Importantly, the penalty from
non-uniform spatial leakage is a function of the subdomain size --- smaller
partitions lead to a larger penalty from the load imbalance. This may limit the
utility of domain decomposition for reactor analysis.

The analysis and results from \autoref{chap:domain-decomp} were published in
\cite{jcp-siegel-2012-2}.

\subsection{Future Work}

In \autoref{chap:domain-decomp}, the assessment of the load imbalance penalty
was done based on measurements of the Monte Carlo performance benchmark using
OpenMC. We suggest that future work look at measurements on an actual reactor
model, such as the MIT PWR benchmark. The enrichment zoning will likely result
in a smaller load imbalance penalty from non-uniform particle densities (since
the power distribution should be flatter), but greater material heterogeneities
could mean that the load imbalance penalty from non-uniform spatial leakage is
worse.

Additionally, while the analysis in \autoref{chap:domain-decomp} looked at the
load imbalance penalty for various domain sizes, it was not mentioned what
domain size will really be necessary for reactor analysis. This will be largely
determined by the tally memory requirements since they will impose a limit on
the size of a spatial subdomain. Other methods proposed in the literature such
as overlapping domains and domain replication should also be accounted for.

Spatial leakage rates may also be sensitive to the use of survival biasing or
other weight adjustment methods used in Monte Carlo eigenvalue
calculations. This effect should be quantified to assess whether this would
substantially change the results we arrived at in \autoref{chap:domain-decomp}.

\section{Tally Memory --- Data Decomposition}

The main alternative to domain decomposition for reducing tally memory
requirements is data decomposition. Although this idea has existed in the
literature for eight years \cite{trans-brown-2004}, little to no work or
analysis has been carried out until now. The last objective of this thesis was
to enhance our understanding of data decomposition algorithms to determine
whether they might enable realistic reactor analysis using Monte Carlo.

\subsection{Contributions of Present Work}

In \autoref{chap:data-decomp}, an algorithm for decomposing large tally data in
Monte Carlo particle transport simulations was proposed, analyzed, and
implemented/tested in OpenMC. The algorithm relies on disjoint sets of compute
processes and servers of which the former simulate particles moving through the
geometry and the latter runs in a continuous loop receiving scores from the
compute processors and incrementing tallies. The analysis in
\autoref{sec:tally-server-analysis} showed that for a range of parameters
relevant to LWR analysis, the tally server algorithm should perform with minimal
overhead on contemporary supercomputers. The implementation of the algorithm in
OpenMC was tested on the Intrepid and Titan supercomputers and was demonstrated
to perform well over a wide range of the parameters. We thus conclude that the
tally server algorithm is a successful approach to circumventing classical
on-node memory constraints en route to unprecedentedly detailed Monte Carlo
reactor simulations.

The work presented on the tally server algorithm has been submitted for
publication in \cite{jcp-romano-2013}.

\subsection{Future Work}

The first implementation of the tally server algorithm in OpenMC relied on
blocking communication semantics. It would be worthwhile to implement
non-blocking communication to further reduce the overhead from the tally server
algorithm. While the algorithm performed well on contemporary supercomputers, on
a machine more within the reach of a typical user, such as a small cluster,
network communication may become a major problem.

One of the important parameters in the theoretical analysis of the tally server
algorithm is the number of events per particle. This parameter will be affected
by the use of survival biasing and other variance reduction techniques. Namely,
survival biasing will increase the number of events per particle. However, the
absolute simulation time will also increase as a result of the use of survival
biasing. In any event, the effect of survival biasing on this parameter and the
feasibility of the tally server algorithm should be quantified.

The number of events per particle will also be affected by the discretization of
a fuel pin into axial and radial segments. Such discretization will be necessary
for detailed depletion calculations. Thus, this effect should also be quantified
and its impact on the overhead assessed.

The tests of the tally server algorithm on Intrepid and Titan were only
performed on up to 1,024 processors. Although the performance showed little
dependence on the number of processors (as predicted), it would nevertheless be
instructive to perform a very large simulation on many processors and with very
large tally requirements.

\section{Other Future Work}

One area that has received, undeservedly, little attention in this thesis is the
coupling of neutronics to other fields such as thermal-hydraulics, transients,
fuel performance, and mechanics. Without feedback from these fields, one can
only obtain a solution to the \emph{wrong} problem. For the purpose of realistic
reactor analysis using Monte Carlo, much work will be needed to best account for
multi-physics feedback. We refer the reader to \cite{net-martin-2012} for a
review of current work in this area.

Neutronic simulations are typically performed assuming no coupling between
neutron and photon physics. As a result, it is necessary to make assumptions
regarding the distribution of photon energy deposition. A typical procedure is
to ``smear'' the photon energy deposition over the problem. Again, to truly
obtain an accurate solution, explicit coupling of neutron and photon transport
is needed.

Many Monte Carlo codes do not account for physics effects that may be important
in certain situations: to name a few, the impact of low-energy resonance
scattering from heavy nuclides \cite{ane-becker-2009, physor-sunny-2012}, the
epithermal scattering of neutrons from Hydrogen via the short collision time
approximation \cite{mc-sutton-2009}, and a continuous $S(\alpha,\beta,T)$
treatment \cite{physor-pavlou-2012}. To the author's knowledge, MC21 is the only
code that can account for all these effects. Wider acceptance and implementation
of these methods will be needed for reactor analysis.

The last issue we will mention is that the memory required to store material
compositions may become non-trivial for depletion calculations. Consider a
reactor with 193 assemblies, 264 pins in each assembly, and a subdivision of
each pin into 100 axial segments and 10 radial rings: a total of about 254
million materials. For a depletion calculation, the densities of up to 300
nuclides would need to be stored for each material. The material compositions
alone would therefore require over 120 GB of memory. This problem would affect
not only Monte Carlo methods but deterministic methods as well.
